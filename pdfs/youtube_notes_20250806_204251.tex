\documentclass[11pt,a4paper]{article}
\usepackage[utf8]{inputenc}
\usepackage[T1]{fontenc}
\usepackage{geometry}
\usepackage{hyperref}
\usepackage{graphicx}
\usepackage{fancyhdr}
\usepackage{listings}
\usepackage{xcolor}
\usepackage{tcolorbox}
\usepackage{amsmath}
\usepackage{amssymb}

% Page setup
\geometry{margin=1in}
\pagestyle{fancy}
\fancyhf{}
\fancyhead[L]{YouTube Notes}
\fancyhead[R]{\today}
\fancyfoot[C]{\thepage}

% Code block styling
\lstset{
    backgroundcolor=\color{gray!10},
    basicstyle=\ttfamily\footnotesize,
    breaklines=true,
    frame=single,
    rulecolor=\color{gray!30}
}

% Quote styling
\newtcolorbox{myquote}{
    colback=blue!5!white,
    colframe=blue!75!black,
    leftrule=3mm
}

\begin{document}

Okay, I am still unable to retrieve the transcript for the provided YouTube video due to an issue with the YouTubeTranscriptApi. I cannot create study notes without the transcript.

\textbf\{Enhanced Notes (Placeholder)\}

Since I cannot access the transcript, I will create a template for how the enhanced notes would look once the transcript is available.

\textbf\{Metadata\}
*   \textbf\{Creation Date:\} 2024-08-06
*   \textbf\{Tags:\} YouTube, Notes, Enhancement, Template

\textbf\{Table of Contents\}

\begin\{itemize\}
\item  [Introduction](\#introduction)
\item  [Key Concepts](\#key-concepts)
\item  [Examples](\#examples)
\item  [Summary](\#summary)
\end\{itemize\}

<a name="introduction"></a>

\subsection\{1. Introduction\}

(This section would contain an introduction to the video's topic based on the transcript.)

<a name="key-concepts"></a>

\subsection\{2. Key Concepts\}

(This section would outline the key concepts discussed in the video, potentially using tables and diagrams.)

\textbf\{Example Table:\}

| Concept        | Definition                                     | Importance                                     |
| -------------- | ---------------------------------------------- | ---------------------------------------------- |
| Concept A      | Definition of Concept A                       | Why Concept A is important                    |
| Concept B      | Definition of Concept B                       | Why Concept B is important                    |

\textbf\{Example Diagram:\}

\texttt\{`\}
┌─────────────┐     ┌─────────────┐     ┌─────────────┐
│   Concept A   │ --> │   Process   │ --> │   Concept B   │
└─────────────┘     └─────────────┘     └─────────────┘
\texttt\{`\}

<a name="examples"></a>

\subsection\{3. Examples\}

(This section would provide examples illustrating the application of the key concepts.)

*   Example 1: (Explanation based on the transcript)
*   Example 2: (Explanation based on the transcript)

<a name="summary"></a>

\subsection\{4. Summary\}

(This section would offer a concise summary of the video's content.)

\textbf\{Quick Reference Card:\}

| Aspect          | Description                                                                  |
| --------------- | ---------------------------------------------------------------------------- |
| Main Topic      | (Main topic of the video)                                                    |
| Key Takeaways   | (List of key takeaways)                                                    |
| Further Reading | (Links to related resources, if mentioned in the video or deemed appropriate) |
\end{document}
